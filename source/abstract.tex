\abstractCZ {
    Tato práce analyzuje a vzlepšuje demonstrátor robotů v pro automatizované sklady využívaný na pracovišti CIIRC.
    Hlavním úkolem této práce je vylepšit stávající funkcionalitu, přidat nové funkcionality a umožnit použítí formátu kompatibilního s {\mapfIR} projektem. 
    Výsledný software by měl zásadně zlepšit dojem z práce s demonstrátorem pro studenty, kteří budou v budoucnu vyvíjet algoritmy pro {\mapfIR} solver. Nedílnou součástí práce je také vylepšení celkového dojmu při prezentaci externím subjektům.
    }
\abstractEN {
    This work analyzes and improves upon the multi-agent planning system demonstrator currently used at the Department of Cybernetics.
    The main objective is to improve existing functionality, add new features and enable the usage of {\mapfIR} format. 
    The final software should significantly improve the developer experience for future students developing algorithms for the {\mapfIR} solver, enhance the impression when presenting to $3^{rd}$ parties, and provide a better user experience overall.
    }
\thanks {
    I would like to thank my supervisor
    {\kulich}, for excellent mentoring and guidance and for providing great advice when I needed it. I would also like to thank {\rybecky} for providing the base software solution and for introducing me to the problem domain. Finally, I
    would like to acknowledge the co-authors of the original Fleet Management System and its components, mainly Ing. Lukáš Bertl and Ing. Jakub Hvězda Ph.D. The original mapfIR project also rightfully deserves appreciation.
    }
\declaration {
    I declare that the presented work was developed independently and that I have listed all sources of information used within it in accordance with the methodical instructions for observing the ethical principles in the preparation of university theses.
    \br \br
    Prague, date 18. May 2020
    \signature
}

\specification {
    As part of the EU solution of the {\safeLog}, a laboratory demonstrator with TurtleBot robots was created for trajectory planning for a group of robots in an automated warehouse. The aim of the thesis is to get acquainted with this environment and develop it further. The final software should enhance the user experience of the previous demonstrator version.
}