\abstractEN {
    This work is analyzing and improving upon the multiagent planning system demonstrator currently used at the Department of Cybernetics.
    The main objective is to improve existing functionality, add new features and enable the usage of {\mapfIR} format. 
    It will greatly improve the developer experience for other students working and running simulations developing algorithms for the {\mapfIR} solver.
    }
\thanks {
    I would really like to thank my supervisor
    {\kulich} for great mentoring and guidance and providing great advice when it was needed. I would also like to thank {\rybecky} for providing the base software solution and for introducing me to the problem domain. Finally, I
    would like to acknowledge the co-authors of the original Fleet Management System and its components, mainly Ing. Lukáš Bertl and Ing. Jakub Hvězda Ph.D. The original mapfIR project also rightfully deserves appreciation.
    }
\declaration {
    I declare that the presented work was developed independently and that I have listed all sources of information used within it in accordance with the methodical instructions for observing the ethical principles in the preparation of university theses.
    \br \br
    Prague, date 18. May 2020
    \signature
}

\specification {
    As part of the EU solution of the SafeLog project, a laboratory demonstrator with TurtleBot robots was created for trajectory planning for a group of robots in an automated warehouse. The aim of the thesis is to get acquainted with this environment and develop it further. The specific procedure is as follows:
    {
        \begitems \style x
        * Get acquainted with the current state of development of the demonstrator and the simulator for multi-agent planning (https://github.com/Kei18/mapf-IR).
        * Modify the simulator to serve as the basic user interface (GUI) of the demonstrator.
        * Display robot positions obtained from the Vicon system in the GUI.
        * Integrate the supplied components for planning and plan execution into the demonstrator.
        * Evaluate experimentally properties of the implemented system. Describe and discuss obtained results.
    \enditems
    }
}