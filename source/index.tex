\chap Úvod


Text úvodu.

\sec Myšlenka

Aktuálně používaný software k multiagentní simulaci umožňuje spouštění předem vygenerovaného plánu na fyzických robotech. Myšlenkou této práce je vylepšit stávající software tak, aby:
\begitems \style n
    * Umožňoval čistě virtuální běh
    * Měl GUI, které zobrazí feedback od robotů.
    * Standardizovat vstupní formát s \mapfIR
\enditems

\sec Original assignment
In colaboration with SafeLog, the laboratory demostrator TurtleBot was created for robot fleet trajectory planning in automated warehouses. The goal of this semestral project is to analyze this framework and improve upon its codebase. The tasks are as follows:

\begitems \style n
    * Analyze multi-agent demonstrator and simulator {\mapfIR} and its current state of development 
    * Edit the simulator to serve as basic user interface for demonstrator (GUI)
    * Zobrazovat pozice robotů získané ze systému Vicon v GUI.
    * Integrovat dodané komponenty pro plánování a generování plánů do systému .
\enditems

\sec Current state
Currently, there exists {\oldRepo} of this project, that enables only basic functionality for the simulations. This maily includes:

\begitems \style n
    * Loading generated simulation data (in {\oldFormat})
    * Sending commands in predefined order to the robots
\enditems

It does the basics necessary to run the simulation. But it does not enable the user to adjust or view the result. The {\oldRepo} receives feedback from robots after command execution. It has no continuous data about the robots real-world positions. If collision happens between two robots, the {\oldRepo} has no way of detecting and adjusting the robots paths.

Current proprietary format is not compatible with {\mapfIR} input nor ouput format, which makes the software at its current state useless for students developing algorithms for the mentioned solver. And great amount of effort is required for students to run a simulation using this software, if they happened to be working with compatible format.

\sec Vision



Aktuálně používaný software k multiagentní simulaci umožňuje spouštění předem vygenerovaného plánu na fyzických robotech. Myšlenkou této práce je vylepšit stávající software tak, aby:
\begitems \style n
    * Umožňoval čistě virtuální běh
    * Měl GUI, které zobrazí feedback od robotů.
    * Standardizovat vstupní formát s \mapfIR
\enditems


% \label[test]
% ~\ref[test]


\bye
\endtt
