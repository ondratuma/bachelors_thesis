\abstractEN {
    This work is analyzing and improving upon the multiagent planning system demonstrator currently used at the Department of Cybernetics.
    The main objective is to improve existing funccionality, add new features and enable the usage of {\mapfIR} format. 
    It will greatly improve developer experience for other students working and running simulations developing alhorithms for the {\mapfIR} solver.
    }
\abstractCZ {
    Tato práce analyzuje možnosti a rozšiřuje demonstrátor multiagentních systémů.
    }
\thanks {
    Chtěl bych poděkovat především vedoucímu práce \kulich za skvělé uvedení do problematiky a za konzultace,
     které byly nezbytně nutné pro dokončení projektu.
    }
\declaration {
    Prohlašuji, že jsem předloženou práci vypracoval samostatná a že jsem uvedl veškeré
použité informační zdroje v souladu s Metodickým pokynem o dodržování etických principů při přípravě vysokoškolských závěrečných prací.
}
\makefront



\chap Úvod


Text úvodu.

\sec Myšlenka

Aktuálně používaný software k multiagentní simulaci umožňuje spouštění předem vygenerovaného plánu na fyzických robotech. Myšlenkou této práce je vylepšit stávající software tak, aby:
\begitems \style n
    * Umožňoval čistě virtuální běh
    * Měl GUI, které zobrazí feedback od robotů.
    * Standardizovat vstupní formát s \mapfIR
\enditems

\sec Zadání práce
V rámci řešení EU projektu SafeLog vznikl laboratorní demonstrátor s roboty TurtleBot pro plánování trajektorií pro skupinu robotů v automatizovaném skladu. V semestrálním projektu je cílem se seznámit s tímto prostředím a toto dále rozvíjet. Konkrétní postup je následující:


\begitems \style n
    * Seznámit se s aktuální stavem vývoje demonstrátoru a se simulátorem pro multi-agentní plánování \mapfIR.
    * Upravit simulátor tak, aby sloužil jako základní uživatelské rozhraní (GUI) demonstrátoru.
    * Zobrazovat pozice robotů získané ze systému Vicon v GUI.
    * Integrovat dodané komponenty pro plánování a generování plánů do systému .
\enditems

\sec Current state
Currently, there exists {\oldRepo} of this project, that enables only basic funncionaly for the simulations. This maily includes:

\begitems \style n
    * Loading generated simulation data (in {\oldFormat})
    * Sending commands in predefined order to the robots
\enditems

It does the basics necessary to run the simulation. But it does not enable the user to adjust or view the result.
The {\oldRepo} receives feedback from robots after command execution. It has no continuous data about the robots real-world positions. If collision happens between two robots, the {\oldRepo} has no way of detecting and adjusting the robots paths

\sec Though
The primary objective of this work is to improve the usability of current software for running simulations, and to improve the capabilities of the software.

\sec Vision

Aktuálně používaný software k multiagentní simulaci umožňuje spouštění předem vygenerovaného plánu na fyzických robotech. Myšlenkou této práce je vylepšit stávající software tak, aby:
\begitems \style n
    * Umožňoval čistě virtuální běh
    * Měl GUI, které zobrazí feedback od robotů.
    * Standardizovat vstupní formát s \mapfIR
\enditems


% \label[test]
% ~\ref[test]


\bye
\endtt
