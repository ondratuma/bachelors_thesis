\chap Introduction


In this chapter, I will be describing the original assignment, the benefits of this project and the current state of the work I will be basing this project upon.

\sec Original assignment
In collaboration with SafeLog, the laboratory demonstrator TurtleBot was created for robot fleet trajectory planning in automated warehouses. The goal of this semestral project is to analyze this framework and improve upon its codebase. The tasks are as follows:

\begitems \style n
    * Analyze multi-agent demonstrator and simulator {\mapfIR} and its current state of development 
    * Edit the simulator to serve as basic user interface for demonstrator (GUI)
    * Display robot positions received from the Vicon system
    * Integrate components for planning and generating plans into the system.
\enditems

\sec Demonstrator current state analysis
Currently, there exists {\oldRepo} of this project, that enables only basic functionality for the simulations. This mainly includes:

\begitems \style n
    * Loading generated simulation data (in {\oldFormat})
    * Run against physical robots in a lab with commands in a predefined order
\enditems

It does the basics necessary to run the simulation. But it does not enable the user to adjust or view the result. The {\oldRepo} receives feedback from robots after command execution. It has no continuous data about the robots real-world positions. If collision happens between two robots, the {\oldRepo} has no way of detecting and adjusting the robots paths.

Current proprietary format is not compatible with {\mapfIR} input nor output format, which makes the software at its current state useless for students developing algorithms for the mentioned solver. And great amount of effort is required for students to run a simulation using this software, if they happened to be working with compatible format.

\sec Demonstrator proprietary format
The format used to represent the map is XML 

\sec End goal

The goal is to create version of the demonstrator that would improve developer experience for developers working on algorithmic solution for {\mapfIR}, and enable the possibility of presenting it to external visitors for representational purposes (ie. possible applicants).

Therefore the department will benefit internally from the increased productivity, and lesser time spent on demonstration setup, and externally when presenting to possible applicants.
It should include the following features:
\begitems \style n
    * Purely virtual run
    * GUI with information about current run
    * Standardize input and output formats with \mapfIR
\enditems


% \label[test]
% ~\ref[test]


\bye
\endtt
