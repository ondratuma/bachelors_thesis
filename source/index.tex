
\worktype [B/CZ]
\faculty {F3}
\department {Katedra kybernetiky}
\title {Demonstrátor systému plánování pro více agentů}
\author {Tůma Ondřej}
\supervisor {\kulich}
\workname {Semestrální projekt}


\date {Prosinec 2021}

\abstractEN {This document is for testing purpose only.}
\abstractCZ {Tato práce analyzuje možnosti a rozšiřuje demonstrátor multiagentních systémů.}
\thanks {Chtěl bych poděkovat především vedoucímu práce \kulich za skvělé uvedení do problematiky a za konzultace, které byly nezbytně nutné pro dokončení projektu.}
\declaration {Prohlašuji, že jsem předloženou práci vypracoval samostatná a že jsem uvedl veškeré
použité informační zdroje v souladu s Metodickým pokynem o dodržování etických principů při přípravě vysokoškolských závěrečných prací.}
\makefront



\chap Úvod

Text úvodu.

\sec Myšlenka

Aktuálně používaný software k multiagentní simulaci umožňuje spouštění předem vygenerovaného plánu na fyzických robotech. Myšlenkou této práce je vylepšit stávající software tak, aby:
\begitems \style n
    * Umožňoval čistě virtuální běh
    * Měl GUI, které zobrazí feedback od robotů.
    * Standardizovat vstupní formát s \mapfIR
\enditems

\sec Zadání práce
V rámci řešení EU projektu SafeLog vznikl laboratorní demonstrátor s roboty TurtleBot pro plánování trajektorií pro skupinu robotů v automatizovaném skladu. V semestrálním projektu je cílem se seznámit s tímto prostředím a toto dále rozvíjet. Konkrétní postup je následující:


\begitems \style n
    * Seznámit se s aktuální stavem vývoje demonstrátoru a se simulátorem pro multi-agentní plánování \mapfIR.
    * Upravit simulátor tak, aby sloužil jako základní uživatelské rozhraní (GUI) demonstrátoru.
    * Zobrazovat pozice robotů získané ze systému Vicon v GUI.
    * Integrovat dodané komponenty pro plánování a generování plánů do systému.
\enditems
\bye
\endtt
