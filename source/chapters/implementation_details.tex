\chap Implementation
In this chapter I will describe my implementation and the particular solutions used to comply with decisions made in chapter ~\ref[chapter_implementation_decisions].

\sec Application architecture
The application is split between server part and frontend part.
Communicating through abstract interface built upon \nanomsg.

Both parts of application have shared utils classes. The main classes used for communication are Messenger (asynchronous handler for inbound/outbound messages) and Message (serializable container), providing easy-to-use interface and abstracting the network layer away from the programmer. The basic hierarchy of used communication classes is below.
\communicationUml

Server and Frontend each have its own instance of Messenger associated with given Message type, communicating through api defined by given messenger. In the case of SychnronizedStateMessenger, the only the latest version of the synchronized state is publically available to the user.

\communicationSequence

\sec GUI application
The GUI is implemented using {\of}. 



