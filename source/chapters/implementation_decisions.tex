\chap Implementation decisions
In this chapter, I will briefly describe the implementation requirements provided and decisions that were necessary before the actual implementation

\sec Current implementation
The {\oldRepo} is using {\cmake} build system. It provides great benefits compared to {\makefile}. Mainly portability across multiple build systems.
The primary decision was that we were to keep this build system for the server part of the application. This will be important later when choosing gui in the section ~\ref[choosing_gui]


\label[choosing_gui]
\sec GUI library considerations 
The primary candidate for GUI library was {\of}. It provides extensive amount of addons ranging greatly in functionality, with the pros being mainly:
\begitems \style n
    * Event loop synchronized with framerate
    * Interacting with the canvas using framework apis
    * Rich set of GUI widgets (buttons, text fields, etc.)
\enditems
with the only downside being it's dependence on \makefile. Therefore we are not able to use the official version with our build system.

The developers are rather defensive about migrating build systems. The recommended way of building {\of} app is to clone the official repository, create app using their template, and extend the makefiles provided by the original repository. In order to be able to use this project with \cmake, it would be necessary to create header map for each component of the framework.

There appear to be a few attempts to do this. Unfortunately, the wast majority of them is out of date and unusable without heavy modifications.

\def \ofxCMake {\ulink[https://github.com/Kei18/mapf-IR]{ofxCMake}}
\def \ofnode {\ulink[https://github.com/Kei18/mapf-IR]{ofnode}}

The first one being \ofxCMake. Its attempt is to provide the package as {\of} addon. Unfortunately, the latest version is from Aug 13, 2017, and there seems to be no active development keeping it in sync. 

Another one being \ofnode, taking different approach. It has its own fork of {\of} repository, with mappings for {\cmake}.
