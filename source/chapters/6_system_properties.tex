\label[system_properties]
\label[chap_6]
\chap Implemented system properties
The main goal of this chapter is to describe and evaluate the properties of the implemented system and compare them with the properties of the previously used system, while pointing out the benefits and discussing potential improvements.

\sec Demonstration setup
This section describes the process before the demonstration can be run. It includes updates from source code, build of the robots, server, set up of the connections and synchronization of all the elements of the demonstration. \br
The setup has to be done before each demonstration run, which in turn limits the development speed when performed inefficiently. Therefore, I attribute a large impact on the perception of the project as a whole to the tooling provided for the developers working with the demonstrator.
\secc Original system
In the original system, the developer was required to manually connect to each robot while typing the robot IP address and local account password for each one. Then login to the VCS, pull the newly updated source code and ensure it builds correctly on the system. This task was time-consuming and did greatly affect the developer experience.
\secc New system
In the new system, the preferred way to run the demonstration is through the shell script "ciirc-exec.sh" in the project root.
Commands interacting with the server are:
\begitems \style x
    * {./ciirc-exec.sh server:init} - performs initial cloning of the repository
    * {./ciirc-exec.sh server:update} - performs hard reset of the tracked brach and pulls the latest changes from the VCS on the server and fixes permissions for executable files.
    * {./ciirc-exec.sh server:build} - builds the software on the server machine.
    * {./ciirc-exec.sh server:start} - starts the server software
    * {./ciirc-exec.sh server:stop} - stops the server software
    * {./ciirc-exec.sh server:ssh} - opens an interactive shell in the root of the project on the server
\enditems

Commands interacting with the robots are:
\begitems \style x
    * {./ciirc-exec.sh robots:ssh <robotId>} - opens an interactive shell in the root of the project on the the robot
    * {./ciirc-exec.sh robots:init <robotId>} - performs initial cloning of the repository
    * {./ciirc-exec.sh robots:update <robotId>} -  performs hard reset of the tracked brach and pulls the latest changes from the VCS on the robot and fixes permissions for executable files.
    * {./ciirc-exec.sh robots:update:all} - executes "./ciirc-exec.sh robots:update <robotId>" for all available robots.
    * {./ciirc-exec.sh robots:inspect <robotId>} - connects to a running tmux session on the robot
    * {./ciirc-exec.sh robots:start <robotId>} - starts the robot software
    * {./ciirc-exec.sh robots:start:all} - performs "./ciirc-exec.sh robots:start <robotId>" for all available robots.
    * {./ciirc-exec.sh robots:stop} - stops the software running on all robots.
\enditems

The script is intended to reduce the time necessary for the setup and automates most of the initial setup tasks that had to be done manually before. \br
Commands that are connecting directly to robots use the mapping defined in "/robot_definitions.txt" and perform actions using the defined credentials. \br The ":update" script first copies the ssh keys used for the git repository to the target machine, removing the need to ever type a password during script execution, while keeping the access restricted.\br
There are additional scripts for verifying the network connectivity, which proved to be very helpful when eliminating possible bug sources that, in many cases, led to the network connectivity. \br
This greatly decreases the developer time needed to review any changes done to the source code. Great developer experience is also ensured by defining the source repository, branch, and local ssh credentials to server and robots in a single place at the start of the script, allowing them to deploy the complete system from different repository/branch easily.

\secc Comparison
The original UX required the user to have extensive knowledge of the system, know the IP addresses of each one of the robots, and it had a significant time impact. The new UX enables anybody with access to the repository and local network to deploy all the robots in under 3 minutes. Single robot testing can be set up in under a minute. That is a huge time saving when compared to the original workflow, where the update would take approximately 10 minutes after getting to know the system and required an additional 2 hours of getting to know the system.
\sec Runtime behavior
\secc Original system
In the original system, the only way for the user to determine what is the current execution state was to look at logs in the server console or connect manually through ssh to the tmux sessions open on each robot. It required the developer to type the password repetitively, and during my time testing, proved to be really discouraging from getting actual work done.
\secc New system
Since the new system has GUI, it provides a great experience to the user. The user can easily change the demonstration parameters, enable/disable particular robots and change demonstration execution algorithms at runtime.

The GUI also provides information about the connection status for the server, and individual robots, further improving the setup process visibility.

Improved is also the developer experience, as it is possible to stop the demonstration after each step, which makes debugging the whole application a lot simpler, should it ever be needed.
\secc Comparison
In the new system, the user can change the parameters of robots, see their individual positions, enable/disable them and see the avoided potential collisions. Since the original system did not have a GUI, there is no point in directly comparing the UX. 
\sec Runtime demonstration impression
When presenting to $3^{rd}$ party, the speed and consistency of setup are even more crucial, as it can have a negative effect on the presentee should anything go wrong. The visualization becomes more important than the runtime properties, as the $3^{rd}$ party is mostly unaware of them and should be hidden when briefly presenting the result of a planner/execution algorithm. When presenting the new system, the continuous feedback from robots, {\vicon}, and visualization of plans provide a great impression and are largely able to mitigate insignificant inconsistencies should they happen during runtime.
\sec Runtime demonstration problems
The whole setup is mainly dependent on the condition of the local network. The network consists of multiple subnets with relatively low-end endpoints, limiting the network throughput and introducing hard-to-debug problems for the system administrators. During testing, there have been found inconsistencies when routing in local subnets. This has been mitigated to a degree by reversion of the connection flow, where each robot now acts as a server from a networking point of view, waiting for connection from the centralized runtime server, avoiding the dependency on a static IP address of the server. Making it possible to run the demonstration from any suitable device connected to the local network.