\chap Implemented system properties
The main goal of this chapter is to describe and evaluate the properties of the implemented system.

\sec Demonstration setup
This section describes the process before the demonstration can be run. It includes updates from source code, building the robots, server, setting up connections and synchronizing all elements of demonstration.
\secc Original system
In the original system, the developer was required to manually connect to each one of the robots, while typing the local account password, login to the VCS, pull the newly updated source code and ensure it builds correctly on the system. This task was time consuming and did greatly affect the developer experience.
\secc New system
In the new system, the preferred way to run the demostration is through the shell script "ciirc-exec.sh" in the project root. The script usage is as follows
\begtt
Commands for server:
./ciirc-exec.sh server:init
./ciirc-exec.sh server:update
./ciirc-exec.sh server:build
./ciirc-exec.sh server:start
./ciirc-exec.sh server:stop
./ciirc-exec.sh server:ssh

----------------------------------
Commands for robots:
./ciirc-exec.sh robots:ssh <robotId>
./ciirc-exec.sh robots:init <robotId>
./ciirc-exec.sh robots:update <robotId>
./ciirc-exec.sh robots:update:all
./ciirc-exec.sh robots:inspect <robotId>
./ciirc-exec.sh robots:start <robotId>
./ciirc-exec.sh robots:start:all
./ciirc-exec.sh robots:stop
\endtt

The script is intended to reduce the time necessary for the setup, and automates most of the initial setup task that had to be done manually before. The :update script command updates the server and robots from git while using the supplied ssh keys, that are first automatically copied to each device. This greatly decreases the developer time needed to review any changes done to the source code. Great developer experience is also ensured by defining the source repository, branch, local ssh credentials to server and robots at single place at the start of the script.

\secc Comparison
The original UX required the user to have extensive knowledge of the system, know the IP addresses of each one of the robots, and had a significant time impact. The new UX enables anybody with access to the repository and local network to deploy the robots in under 3 minutes.
\sec Runtime behavior
\secc Original system
In the original system, the only was for user to determine what is the current execution state was to look at logs in the server console, or connect manually through ssh to the tmux sessions open on each robot.
\secc New system
Since the new system has GUI, it provides great experience to the user. The user can easily change the demonstration parameters, enable/disable particular robots and change demonstration execution algorithms at runtime.
\secc Comparison
In the new system, the user can change the parameters of robots, see their individual positions, enable/disable them and see the avoided potential collisions. Since the original system did not have a GUI, there is no point in directly comparing the UX. 
\sec Runtime demonstration impression
When presenting to 3rd party, the speed and consistency of setup is even more crutial, as it can have negative effect on the presentee, should anything go wrong. When presenting the new system, the continuous feedback from robots, vicon and visualization of plans provide great impression.
\sec Runtime demonstration problems