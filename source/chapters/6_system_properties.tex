\chap Implemented system properties
The main goal of this chapter is to describe and evaluate the properties of the implemented system.

\sec Demonstration setup
This section describes the process before the demonstration can be run. It includes updates from source code, building the robots, server, setting up connections and synchronizing all elements of demonstration.
\secc New system
In the new system, the preferred way to run the demostration is through the shell script "ciirc-exec.sh" in the project root. The script usage is as follows
\begtt
Commands for server:
./ciirc-exec.sh server:init
./ciirc-exec.sh server:update
./ciirc-exec.sh server:build
./ciirc-exec.sh server:start
./ciirc-exec.sh server:stop
./ciirc-exec.sh server:ssh

----------------------------------
Commands for robots:
./ciirc-exec.sh robots:ssh <robotId>
./ciirc-exec.sh robots:init <robotId>
./ciirc-exec.sh robots:update <robotId>
./ciirc-exec.sh robots:update:all
./ciirc-exec.sh robots:inspect <robotId>
./ciirc-exec.sh robots:start <robotId>
./ciirc-exec.sh robots:start:all
./ciirc-exec.sh robots:stop
\endtt

The script is intended to reduce the time necessary for the setup, and automates most of the initial setup task that had to be done manually before.

\sec Demonstration setup user experience
\sec Runtime behavior
\sec Runtime user experience
\sec Runtime demonstration impression
\sec Runtime demonstration problems