\label[chapter_implementation_decisions]
\chap Implementation decisions
In this chapter, I will briefly describe the implementation requirements provided and decisions that were necessary before the actual implementation

\sec Compatibility considerations
The {\oldRepo} is using {\cmake} build system. It provides great benefits compared to {\makefile}. Mainly portability across multiple build systems.
The primary decision was that we were to keep this build system for the server part of the application. This will be important later when choosing GUI in the section ~\ref[choosing_gui]


\label[choosing_gui]
\sec GUI library considerations 
The primary candidate for the GUI library was {\of}. It provides an extensive amount of addons ranging greatly in functionality, with the pros being mainly:
\begitems \style n
    * Event loop synchronized with framerate
    * Interacting with the canvas using framework APIs
    * Rich set of GUI widgets (buttons, text fields, etc.)
\enditems
with the only downside being its dependence on the \makefile. Therefore we are not able to use the official version with our build system.

The developers are rather defensive about migrating build systems. The recommended way of building {\of} app is to clone the official repository, create an app using their template, and extend the makefiles provided by the original repository. In order to be able to use this project with \cmake, it would be necessary to fork the original repository, create a header map for each component of the framework, potentially breaking support for unported addons.
This would break the main benefit of the framework, extensibility, while the time investment necessary would certainly overrun the budget for this thesis.

\hfill \break
There appear to be a few attempts to do this. Unfortunately, the vast majority of them is out of date and unusable without heavy modifications. During the preparation for this thesis, I tried to setup all of them, without success, as they are out of date. Below are few example approaches to this problem.



The first one being \ofxCMake. It attempts to provide the package as {\of} addon. Unfortunately, the latest version is from Aug 13, 2017, and there seems to be no active development keeping it in sync. 

Another one being {\ofnode}, taking a different approach. It has its own fork of the whole {\of} repository, with mappings for {\cmake}. The fork approach introduces inevitable split from the main branch, implicitly keeping it out of sync with no support for new features. \br
The latest version is from on Nov 7, 2018. I ran into several issues with system libraries with this version and was not able to compile even the bare repository.\br\br
From the forums, it seems that both approaches had been working for at least few users for some time. The main issue with both approaches seems to be the development outside of the main repository, that led to their deprecation, as the developers had to maintain it themselves, what both of them stopped doing.
It was then decided, that it would be most beneficial to split the GUI into separate application and keep using the framework as intended by the authors, as the approach to port it to {\cmake} had been tried before, and as we can see, it did not work in the long term.\br\br
The implementation will be, however, done with the possible switch to single application in mind, trying to abstract most of the communication away from the developer, making the switch relatively easy, were the developers ever to introduce {\cmake} support.

\sec GUI implementation considerations
First aspect was speed and network bandwidth. The simulation might run hundreds of robots, depending on the simulation. So on each robot move, it is necessary to send the minimal amount of information required in order to keep the latency to a minimum. 
It was then decided, that the plan will be synchronized between both applications on simulation start or plan change. Any messages regarding the robot positions will be strictly referring to the synchronized plan.

Second aspect was ease-of-development. It is necessary to create enough abstraction for the developer not to have to worry about the network. A set of abstractions allowing to share messages and state in asynchronous manner will be necessary. Ideally, the developer should not care whether the apps are running in different threads or processes, the api should remain the same.

\sec Virtual run implementation considerations
In the original demonstrator, there was no support for the virtual run. It is, however, truly inconvenient, having to have physical robots in order to evaluate execution plan. Virtual run will also be useful for testing the server and frontend. For those cases, the server and frontend should not be able to distinguish between real robots, and virtual run, as we should not introduce any code related to testing to the tested parts of application. Given these constraints, the virtual run will be implemented using "virtualized robots", that will be listening and responding on the same ports, as real robots would. This approach should make both use-cases as efficient, as real run, while keeping test related code to a minimum.

\sec Coordinate system convention
The coordinate system of the {\mapfIR} project is perceived to grow from the upper left corner, as described in {~\ref[mapfir_coordinate_system]}, while the coordinate system of the original system grows from the lower left corner, as described here {~\ref[fc_coordinate_system]}. For the convenience of the user, it is best to keep the orientations of all coordinate systems used within the new demonstrator in the same direction. Since the robots and vicon coordinate systems are tied to the physical equipment in the laboratory, the new system will comply with the convention used by the original system. \br
In the {\mapfIR} project, the only place where the rotation of the coordinate system should matter is during the visualization, as the solver and visualizer use the same convention. This means, that the perception of the grid coordinate system will be rotated when visualized on the {\mapfIR} visualizer and the new demonstrator. This might cause minor inconvenience for the developers working with both projects, but is necessary to keep all the components in the new system consistent without the need to perform rotations and confusing the developer while working with the software.