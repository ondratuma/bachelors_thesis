\chap Introduction
With the increased demand in the logistics industry in recent years, new solutions enabling higher efficiency and accuracy are necessary. Autonomous robot fleet systems are at the frontier of used solutions, enabling logistics providers to operate at a much larger scale. New algorithms are constantly being developed for this purpose. With that comes the necessity to test them and evaluate their properties. The real-world properties of such algorithms might not be obvious when they are developed in the academic environment. It is, therefore, necessary to create conditions similar to the ones in real warehouses in order to design the best algorithms that will be able to avoid situations that might arise only due to otherwise unexpected causes.
\br
\br
The goal of this thesis is to improve the software for autonomous warehouse demonstrations currently used by the Intelligent and Mobile Robotics Group (IMR) at Czech Institute of Informatics, Robotics and Cybernetics (CIIRC). 

\sec Current state
Currently, the IMR is operating a demonstrator of autonomous warehouse robots. 
Its main purpose is the execution of a solution to a pathfinding problem. The problem usually consists of several agents with assigned starting points and a set of goals that the agents have to arrive to. The solution is then created, assigning each agent a path to a specific goal.
The demonstrator is part of the {\oldRepo}. The version currently used is hard to set up, supports only a proprietary format, and creates additional friction in the workflow. Moreover, the demonstrator output is text-only, making it generally unappealing to work with. It enables only basic functionality for the simulations, which mainly include:

\begitems \style n
    * Loading generated simulation data (in the {\oldFormat})
    * Run against physical robots in a lab with commands in a predefined order
\enditems

It does the basics necessary to run the simulation, but it does not enable the user to adjust or view the result. The information flow is limited, as the only information relevant to the execution is the feedback from robots after command execution. It has no continuous data about real-world positions. If a collision happens between two robots, the executor has no way of detecting and adjusting the robot's paths.

The IMR also experiments with the {\mapfIR} project \cite[mapfir_repo], a software-only solution for the generation and visualization of similar plans.
The current data representation format used by the demonstrator is XML-based. It is not compatible with {\mapfIR}'s input nor output format, which makes the software at its current state useless for students developing algorithms for the {\mapfIR} solver. 

Another disadvantage is the substantial amount of work and knowledge required to set up a basic demonstration run. This substantially limits possible use-cases for this project, as the setup takes significantly more time than the actual demonstration run. By making the setup process more straightforward, the project could be used more often, increasing its perceived usefulness and likely increasing its chances of getting future updates.

\sec End goal

The goal is to create a version of the demonstrator that would simplify the setup process, improve the experience for developers working on an algorithmic solution with the {\mapfIR} format, allowing them to switch between the original visualizer and this demonstrator seamlessly, and enable the possibility of presenting it to external visitors for representational purposes (i.e., possible applicants) by enhancing the user interface.\br
The Graphical User Interface (GUI) should enable the user to set up demonstration settings and interact with the system during the demonstration run.
Among other goals is the incorporation of the {\vicon} system, allowing for real-time tracking of robots. Another option available for getting real-time positions that will be used is the feedback from the robots themselves.\br
\br

The department will benefit both internally and externally from this work. The internal benefits are mainly increased productivity of future developers working on this project and lesser time spent on the demonstration setup. The external benefits include the additional possibility of presenting to future applicants.
The complete list of features should include:
\begitems \style n
    * Simplified demonstration setup
    * Purely virtual run
    * GUI with information about the current run and the ability to modify the run, such as disabling individual agents. 
    * Standardize input and output formats with \mapfIR
    * {\vicon} system integration
    * Ability to display real-time robot positions ({\vicon}/robots)
    * Easy integration of future changes to the demonstration execution
\enditems
\br
The rest of the text is as follows.
{\link[ref:chap_2]{\Black}{\it Chapter 2}} analyzes the {\mapfIR} project to help us understand the formats it uses and simplify further development. {\link[ref:chap_3]{\Black}{\it Chapter 3}} analyzes the current state of the software used as a basic prerequisite for its enhancement. {\link[ref:chap_4]{\Black}{\it Chapter 4}} goes over implementation decisions for the creation of the final software, while {\link[ref:chap_5]{\Black}{\it Chapter 5}} describes the actual implementation details. {\link[ref:chap_6]{\Black}{\it Chapter 6}} sums up the workflow and user experience and compares the original and new software. {\link[ref:chap_7]{\Black}{\it Chapter 7}} serves as the summary of this thesis.