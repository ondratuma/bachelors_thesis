\chap Introduction
The general motivation for this thesis is the development of autonomous systems, planning algorithms, the testing necessary to evaluate their properties in real conditions and their demonstration to interested parties.

\sec The goal
The goal of this thesis is to improve the software for autonomous warehouse demonstrations currently used by the The Intelligent and Mobile Robotics Group (IMR) as CIIRC. The version currently used is hard to setup, supports only proprietary format, and creates additional friction in the workflow. The new software should eliminate most of the friction points in the development process, decreasing the overhead for demonstration setup, and further decreasing the barrier of entry for new developers interested in the project. It should also improve the appeal of the software by adding Graphical User Interface (GUI). Among other desired improvements is the switch to a format used by the {\mapfIR} project, that serves as solver and virtual-only visualizer for similar use-cases.

\sec Current state
Currently, the The Intelligent and Mobile Robotics Group (IMR) at CIIRC is operating a demonstrator of autonomous warehouse robots. The demonstrator is part of the {\oldRepo} that enables only basic functionality for the simulations, which mainly include:

\begitems \style n
    * Loading generated simulation data (in the {\oldFormat})
    * Run against physical robots in a lab with commands in a predefined order
\enditems

It does the basics necessary to run the simulation. But it does not enable the user to adjust or view the result. The {\oldRepo} receives feedback from robots after command execution. It has no continuous data about real-world positions. If a collision happens between two robots, the executor has no way of detecting and adjusting the robot's paths.

The current data representation format is XML-based. It is not compatible with {\mapfIR} input nor output format, which makes the software at its current state useless for students developing algorithms for the {\mapfIR} solver. 

Another disadvantage is the substantial amount of work and knowledge required to set up a basic demonstration run. This substantially limits the possible use-cases for this project, as the setup takes significantly more time than the actual demonstration run. By making the setup process more straightforward, the project could be used more often, increasing its perceived usefulness and likely increasing its chances of getting future updates.

\sec End goal

The goal is to create a version of the demonstrator that would simplify the setup process, improve the experience for developers working on an algorithmic solution with the {\mapfIR} format, allowing them to switch between the original visualizer and this demonstrator seamlessly and enable the possibility of presenting it to external visitors for representational purposes (i.e. possible applicants) by enhancing the user interface.\br
Another significant goal is the introduction of GUI, which would enable the user to set up demonstration settings and interact with the system during the demonstration run.
Among other goals is the incorporation of the {\vicon} system, allowing for real-time tracking of robots. Another option available for getting real-time positions that will be used is the feedback from the robots themselves.\br
\br
Therefore the department will benefit both internally and externally from this work. The internal benefits are mainly increased productivity of future developers working on this project and lesser time spent on the demonstration setup. The external benefits include the additional possibility of presenting to future applicants.
The complete list of features should include:
\begitems \style n
    * Simplified demonstration setup
    * Purely virtual run
    * GUI with information about the current run and the ability to modify the run, such as disabling individual agents. 
    * Standardize input and output formats with \mapfIR
    * {\vicon} system integration
    * Ability to display real-time robot positions ({\vicon}/robots)
    * Easy integration of future changes to the demonstration execution
\enditems