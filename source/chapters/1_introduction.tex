\chap Introduction

This chapter describes the original assignment, the motivation behind my work, the benefits of this project, and the current state of the work I will be basing this project upon.

\sec Original assignment
As part of the EU solution of the {\safeLog} project, a laboratory demonstrator with TurtleBot robots was created for trajectory planning for a group of robots in an automated warehouse. The aim of the thesis is to get acquainted with this environment and develop it further. The specific procedure is as follows:
\begitems \style n
    * Get acquainted with the current state of development of the demonstrator and the simulator for multi-agent planning (https://github.com/Kei18/mapf-IR).
    * Modify the simulator to serve as the basic user interface (GUI) of the demonstrator.
    * Display robot positions obtained from the {\vicon} system in the GUI.
    * Integrate the supplied components for planning and plan execution into the demonstrator.
    * Evaluate experimentally properties of the implemented system. Describe and discuss obtained results.
\enditems

\sec Current state
Currently, there exists the {\oldRepo}, that enables only basic functionality for the simulations. This mainly includes:

\begitems \style n
    * Loading generated simulation data (in {\oldFormat})
    * Run against physical robots in a lab with commands in a predefined order
\enditems

It does the basics necessary to run the simulation. But it does not enable the user to adjust or view the result. The {\oldRepo} receives feedback from robots after command execution. It has no continuous data about real-world positions. If a collision happens between two robots, the executor has no way of detecting and adjusting the robot's paths.

Current data representation format is XML based. It is not compatible with {\mapfIR} input nor output format, which makes the software at its current state useless for students developing algorithms for the {\mapfIR} solver. 

Another disadvantage is substantial amount of work and knowledge required to setup basic demonstration run. This greatly decreases the possible use-cases for this project, as the setup takes substantially more time than the actual demonstration run. By making the setup process simpler, the project could be used more often, increasing it's perceived usefulness, likely increasing it's chances to of getting future updates.

\sec End goal

The goal is to create a version of the demonstrator that would simplify the setup process, improve the experience for developers working on algorithmic solution with the {\mapfIR} format, allowing them to seamlessly switch between the original visualizer and this demonstrator, and enable the possibility of presenting it to external visitors for representational purposes (ie. possible applicants) by enhancing the user interface.\br
Another significant goal is the introduction of GUI, that would enable the user to setup demonstration settings, and to interact with the system during the demonstration run.
Among other goals is the incorporation of the {\vicon} system, allowing for real-time tracking of robots. Another option available for getting real-time positions that will be used is the feedback from the robots themselves.\br
\br
Therefore the department will benefit both internally and externally from this work. Among the internal benefits is mainly increased productivity of future developers working on this project, and lesser time spent on demonstration setup. External benefits include additional possibility to present to future applicants.
The complete list of features should include:
\begitems \style n
    * Simplified demonstration setup
    * Purely virtual run
    * GUI with information about the current run and ability to modify the run, such as disabling individual agents. 
    * Standardize input and output formats with \mapfIR
    * {\vicon} system integration
    * Ability to display real-time robot positions ({\vicon}/robots)
    * Easy integration of future changes to the demonstration execution
\enditems