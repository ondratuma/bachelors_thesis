\chap Introduction

This chapter describes the original assignment, the motivation behind my work, the benefits of this project, and the current state of the work I will be basing this project upon.

\sec Original assignment
As part of the EU solution of the SafeLog project, a laboratory demonstrator with TurtleBot robots was created for trajectory planning for a group of robots in an automated warehouse. The aim of the thesis is to get acquainted with this environment and develop it further. The specific procedure is as follows:
\begitems \style n
    * Get acquainted with the current state of development of the demonstrator and the simulator for multi-agent planning (https://github.com/Kei18/mapf-IR).
    * Modify the simulator to serve as the basic user interface (GUI) of the demonstrator.
    * Display robot positions obtained from the Vicon system in the GUI.
    * Integrate the supplied components for planning and plan execution into the demonstrator.
    * Evaluate experimentally properties of the implemented system. Describe and discuss obtained results.
\enditems

\sec Demonstrator current state analysis
Currently, there exists an {\oldRepo} of this project, that enables only basic functionality for the simulations. This mainly includes:

\begitems \style n
    * Loading generated simulation data (in {\oldFormat})
    * Run against physical robots in a lab with commands in a predefined order
\enditems

It does the basics necessary to run the simulation. But it does not enable the user to adjust or view the result. The {\oldRepo} receives feedback from robots after command execution. It has no continuous data about the real-world positions. If a collision happens between two robots, the {\oldRepo} has no way of detecting and adjusting the robot's paths.

The current proprietary format is not compatible with {\mapfIR} input nor output format, which makes the software at its current state useless for students developing algorithms for the mentioned solver. And a great amount of effort is required for students to run a simulation using this software, if they happened to be working with a compatible format.

\sec End goal

The goal is to create a version of the demonstrator that would improve the developer experience for developers working on algorithmic solution with the {\mapfIR} format, and enable the possibility of presenting it to external visitors for representational purposes (ie. possible applicants).

Therefore the department will benefit internally from the increased productivity, and lesser time spent on demonstration setup, and externally when presenting to possible applicants.
It should include the following features:
\begitems \style n
    * Purely virtual run
    * GUI with information about the current run and ability to modify the run, such as disabling individual agents, recalculating their paths if necessary, and visualizing additional relevant information.
    * Standardize input and output formats with \mapfIR
\enditems