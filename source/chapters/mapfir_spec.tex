\chap A look at mapf-IR
In this chapter I will take a detailed look at the inner workings of \mapfIR project.

{\mapfIR} is {\uv{A simulator and visualizer of Multi-Agent Path Finding (MAPF), used in a paper "Iterative Refinement for Real-Time Multi-Robot Path Planning" (to appear at IROS-21). It is written in C++(17) with CMake (≥v3.16) build. The repository uses Google Test and the original library for 2D pathfinding as git submodules. The visualizer uses openFrameworks and works only on macOS.
    }\cite[MAPFIR:REPO_DESCRIPTION]
}

\sec Introduction
{\mapfIR} consists of 2 parts. A solver and a visualizer. 
The solver is standalone \cmake application and the visualizer is using openFrameworks as in our case. This split is necessary as described in {~\ref[choosing_gui]}.

\label[mapfir_solver]
\sec The solver
\secc{input format}
\mapfirInputFormat 
The input format is as illustrated on ~\ref[mapfir_input_format] starting by name of map file on the first line, followed by number of agents on the second, followed by the seed of the problem, followed by two time configuration options.
The next \uv{number of agents} lines are in the format sX sY gX gY where sX and sY are the x and y coordinates of the starting point, respectively, and gX, gY are the x and y coordinates of the goal, respectively.

\secc{output format}
\mapfirOutputFormat 
The output format is as illustrated on ~\ref[mapfir_output_format]

\sec The visualizer
The visualizer uses output of {~\ref[mapfir_solver]} to visualize the run of generated simulation. It lets the user change the simulations speed and move through discrete time steps. Other features include selecting only single agent to display, stopping agent during simulation and visual features such as hiding/showing of lines to goals.
\mapfirGui 