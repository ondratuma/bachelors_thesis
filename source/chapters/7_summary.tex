\chap Summary
The aim of this thesis was to improve upon the previous version of the FleetControl demonstrator and achieve compatibility with the format used by the {\mapfIR} project. The work was structured as follows.\br
At first, the {\mapfIR} project was analysed in order to identify key features, and decide which ones are desirable in the final software. Afterwards, the {\oldRepo} was reviewed to determine the current state of development and provide us with the ability to determine the path to the final goal. Then, the  software design and decisions for implementation were discussed. The final stage of the project was the implementation itself, where the core features were implemented first, followed by the non crucial ones.\br\br
The evaluation metrics for the project are the factual criteria given in the assignment. Particularly 
\begitems \style n
    * Get acquainted with the current state of development of the demonstrator and the simulator for multi-agent planning (https://github.com/Kei18/mapf-IR).
    * Modify the simulator to serve as the basic user interface (GUI) of the demonstrator.
    * Display robot positions obtained from the {\vicon} system in the GUI.
    * Integrate the supplied components for planning and plan execution into the demonstrator.
    * Evaluate experimentally properties of the implemented system. Describe and discuss obtained results.
\enditems

I am glad to conclude, that the criteria mentioned above were successfully fulfilled, as described in the previous chapters.\br\br
Furthermore, I'd like to introduce one additional criteria that was not mentioned in the assignment, that is, however, quite important when the main outcome is a software. This criteria being the user experience of the software and relevant processes.
The user experience has been discussed in the chapter {\ref[system_properties]}, and based on the outcome, it is concluded to be more than satisfactory, enhancing user experience greatly when compared to the original project.\br
This project is therefore usable for the required use-case, and should allow for simple modifications for future ones.
The future steps will probably include addition of new synchronization algorithms(\ref[synchronization_strategy]) and introduction of support for in-app rerouting, incorporating selected algorithms supported by {\mapfIR} and enabling easy addition and development of custom algorithms. 

 