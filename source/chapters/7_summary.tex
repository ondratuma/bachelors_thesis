\chap Summary
The aim of this thesis was to improve upon the previous version of the FleetControl demonstrator and achieve compatibility with the format used by the {\mapfIR} project. The work was structured as follows.\br
At first, the {\mapfIR} project was analysed in order to identify key features, and decide which ones are desirable in the final software. Afterwards, the {\oldRepo} was reviewed to determine the current state of development and the actual usage. Then, the  software design and decisions for implementation were discussed. The final stage of the project was the implementation itself, where the core features were implemented first, followed by the non crucial ones.\br\br
The evaluation metrics for the project are the factual criteria given in the assignment. Particularly 
\begitems \style n
    * Get acquainted with the current state of development of the demonstrator and the simulator for multi-agent planning (https://github.com/Kei18/mapf-IR).
    * Modify the simulator to serve as the basic user interface (GUI) of the demonstrator.
    * Display robot positions obtained from the {\vicon} system in the GUI.
    * Integrate the supplied components for planning and plan execution into the demonstrator.
    * Evaluate experimentally properties of the implemented system. Describe and discuss obtained results.
\enditems

Given that the system performs the tasks as expected, I am glad to conclude, that the criteria mentioned above were successfully fulfilled.\br\br
Furthermore, I'd like to point out one aspect of the final software, that has not been introduced as evaluation criteria. The actual user experience of the software and relevant processes.
This topic has been discussed in the chapter {\ref[system_properties]}, and it has been greatly enhanced when compared to the original project.\br
This project is therefore usable for the required use-case, and should allow for simple modifications for future ones.
The future steps will probably include addition of new synchronization algorithms(\ref[synchronization_strategy]) and introduction of support for in-app rerouting, incorporating selected algorithms supported by {\mapfIR} and enabling easy addition and development of custom algorithms. 

 