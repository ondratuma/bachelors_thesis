\chap Summary
The aim of this thesis was to improve upon the previous version of the FleetControl demonstrator and achieve compatibility with the format used by the {\mapfIR} project. The work was structured as follows.\br
At first, the {\mapfIR} project was analyzed in order to identify key features and decide which ones are desirable in the final software. Afterward, the {\oldRepo} was reviewed to determine the current state of development and the actual usage. Then, the  software design and decisions for implementation were discussed. The final stage of the project was the implementation itself, where the core features were implemented first, followed by the non-crucial ones.\br\br
The evaluation metrics for the project are the factual criteria given in the assignment. Particularly 
\begitems \style n
    * Get acquainted with the current state of development of the demonstrator and the simulator for multi-agent planning (https://github.com/Kei18/mapf-IR).
    * Modify the simulator to serve as the basic user interface (GUI) of the demonstrator.
    * Display robot positions obtained from the {\vicon} system in the GUI.
    * Integrate the supplied components for planning and plan execution into the demonstrator.
    * Evaluate experimentally properties of the implemented system. Describe and discuss obtained results.
\enditems

Given that the system performs the tasks as expected, I am glad to conclude, that the criteria mentioned above were successfully fulfilled.\br\br

Furthermore, I'd like to point out one aspect of the final software, that has not been introduced as evaluation criteria but proved to be very important during my encounter with the original software. It is the actual user experience of the software and relevant processes. This topic has been discussed in the chapter {\ref[system_properties]}, and it has been greatly enhanced when compared to the original project. It plays a significant role for new developers and might greatly increase the motivation to continue upon the outcome of my work. \br
Future steps will probably include addition of new synchronization algorithms (\ref[synchronization_strategy]) and introduction of in-app rerouting, incorporating selected algorithms supported by {\mapfIR} and enabling easy addition and development of custom algorithms. 
This project, as designed, expects additional modifications to happen in future and should make them as easy as possible.

 